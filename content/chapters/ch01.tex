\section{Grundlagen}



\subsection{Komplexe Zahlen}

\begin{bemerkung}{Komplexe Zahlen}{S. 7}
  \begriff{Imaginäre Einheit}:
  \begin{align}
    i^2 = -1
  \end{align}
  Alle Zahlen der Form
  \begin{align}
    z = x + iy, \quad x,y \in \R
  \end{align}
  bilden die Menge
  \begin{align}
    \label{eq:1_2}
    \C \coloneqq \{ z = x + i \cdot y :\ x,y \in \R \}
  \end{align}
  \begriff{Realteil} von $z$:
  \begin{align}
    \operatorname{Re}(z) \coloneqq x \in \R
  \end{align}
  \begriff{Imaginärteil} von $z$:
  \begin{align}
    \operatorname{Im}(z) \coloneqq y \in \R
  \end{align}
\end{bemerkung}

\begin{bemerkung}{Addition}{S. 8}
  Sei $z = x + i \, y \in \C$ und $w = u + i \, v \in \C$.
  \begin{align}
    \label{eq:addition}
    z + w \coloneqq (x + u) + i \, (y + v) \in \C
  \end{align}
\end{bemerkung}

\begin{bemerkung}{Inverses Element bzgl. Addition}{S. 8}
  Sei $z = x + i \, y \in \C$.
  \begin{align}
    \label{eq:inv_addition}
    -z = (-x) + i \, (-y) = - x - i \, y \in \C
  \end{align}
\end{bemerkung}

\begin{bemerkung}{Multiplikation}{S. 8}
  Sei $z = x + i \, y \in \C$ und $w = u + i \, v \in \C$.
  \begin{align}
    \label{eq:multiplikation}
    z \cdot w
      \coloneqq (x + i \, y) \cdot (u + i \, v)
      = (x \, u - y \, v) + i \, (x \, v + y \, u)
      \in \C
  \end{align}
\end{bemerkung}

\begin{bemerkung}{Inverses Element bzgl. Multiplikation}{S. 8}
  Sei $z = x + i \, y \in \C, z \neq 0$.
  \begin{align}
    \label{eq:inv_multiplikation}
    \frac{1}{z}
      &= \frac{1}{(x + i \, y)}
       = \frac{x - i \, y}{(x + i \, y) \, (x - i \, y)}
       % = \frac{x - i \, y}{x^2 - (i \, y)^2}
       = \frac{x - i \, y}{x^2 - i^2 \, y^2}
       = \frac{x - i \, y}{x^2 + y^2}
        \notag\\
      &= \frac{x}{x^2 + y^2} - i \cdot \frac{y}{x^2 + y^2} \in \C
  \end{align}
\end{bemerkung}

\begin{satz}[Körpereigenschaften von $\C$]{S. 8}
  Die Menge $\C$ der komplexen Zahlen bilden mit oben definierter Addition \eqref{eq:addition} bzw. Multiplikation \eqref{eq:multiplikation} einen Körper.
  Das Einselement dieses Körpers ist
  \begin{align}
    1 + 0 \cdot i = 1 \in \C
  \end{align}
  und das Nullelement ist
  \begin{align}
    z = 0 + 0 \cdot i = 0 \in \C .
  \end{align}
  Es gelten somit die Rechenregeln (für $z, v, w \in \C$; $z = x + i \cdot y$):
  \newcommand{\tabone}{39mm}
  \begin{enumerate}[label=\alph*)]
    \item $(z + v) + w = z + (v + w)$ \tabto{\tabone} Assioziativgesetz
    \item $z + 0 = z$ \tabto{\tabone} neutrales Element
    \item $z + (-z) = 0$ \tabto{\tabone} inverses Element; wobei \eqref{eq:inv_addition} gilt
    \item $z + w = w + z$ \tabto{\tabone} Kommutativgesetz
    \item $(z \cdot v) \cdot w = z \cdot (v \cdot v)$ \tabto{\tabone} Assoziativgesetz
    \item $z \cdot 1 = z$ \tabto{\tabone}  neutrales Element
    \item $\displaystyle z \cdot \frac{1}{z} = 1$ für $z \neq 0$ \tabto{\tabone} inverses Element; wobei \eqref{eq:inv_multiplikation} gilt
    \item $z \cdot w = w \cdot z$ \tabto{\tabone} Kommutativgesetz
    \item $z \cdot (v + w) = z \cdot v + z \cdot w$ \tabto{\tabone} Distributivgesetz
  \end{enumerate}
\end{satz}

\begin{bemerkung}{Binomische Formel}{S. 10}
  Mit $z, w \in \C$ und $n \in \N$:
  \begin{align}
    (z + w)^n
      = \sum_{k=0}^n \binom{n}{k} \cdot z^k \cdot w^{n-k}
      = \sum_{k=0}^n \frac{n!}{k! \, (n-k)!} \cdot z^k \cdot w^{n-k}
  \end{align}
\end{bemerkung}

\begin{definition}[Absolutbetrag, konjugiert komplexe Zahl]{S. 10}
  \label{def:absolutbetrag}
  Es sei $z = x + i \cdot y = (x, y) \in \C$.
  Dann ist
  \[ |z| \coloneqq \sqrt{x^2 + y^2} \geq 0 \in \R \]
  der \begriff{Absolutbetrag} von $z$ und
  \[ \overline{z} \coloneqq x - i \cdot y \]
  die zu $z$ \begriff{konjugiert komplexe Zahl}.
\end{definition}

\begin{bemerkung}{Absolutbetrag mit konjugiert komplexer Zahl}{S. 11}
  Wegen $z \cdot \overline{z} = (x + i \, y) \, (x - i \, y) = x^2 - i^2 y^2 = x^2 + y^2 = |z|^2$ folgt:
  \begin{align}
    |z| = \sqrt{z \cdot \overline{z}}
  \end{align}
\end{bemerkung}

\begin{bemerkung}{Abstand komplexer Zahlen}{S. 11}
  Sei $z = x + i \, y \in \C$ und $w = u + i \, v \in \C$.
  Dann ist ihr Abstand:
  \begin{align}
    | z - w |
      = | (x-u) + i \, (y-v) |
      \stackrel{\text{\tiny{Def. \ref{def:absolutbetrag}}}}{=} \sqrt{(x-u)^2 + (y-v)^2 }
  \end{align}
\end{bemerkung}

\begin{satz}[Rechenregeln]{S. 11}
  Für $z = x + i \, y$, $w = u + i \, v$ gelten folgende Rechenregeln:
  \begin{enumerate}
    \item $\displaystyle \overline{z \pm w} = \overline{z} \pm \overline{w}$
    \item $\displaystyle \overline{z \cdot w} = \overline{z} \cdot \overline{w}$
    \item $\displaystyle \overline{\left( \frac{z}{w} \right)} = \frac{\overline{z}}{\overline{w}}$ $(w \neq 0)$
    \item $\displaystyle \overline{\left( \overline{z} \right)} = \overline{z}$
    \item $\displaystyle \left| \overline{z} \right| = \left| z \right|$
    \item $\displaystyle \operatorname{Re}(z) = \frac{z + \overline{z}}{2}$\\
      $\displaystyle \operatorname{Im}(z) = \frac{z - \overline{z}}{2}$
    \item $\displaystyle \operatorname{Re}(z_1 + \dots + z_n) = \operatorname{Re}(z_1) + \dots + \operatorname{Re}(z_n)$\\
      $\displaystyle \operatorname{Im}(z_1 + \dots + z_n) = \operatorname{Im}(z_1) + \dots + \operatorname{Im}(z_n)$
    \item $\displaystyle \operatorname{Re}(z) \leq |z|$\\
      $\displaystyle \operatorname{Im}(z) \leq |z|$
    \item $\displaystyle | z \cdot w | = |z| \cdot |w|$\\
      $\displaystyle \left| \frac{z}{w} \right| = \frac{|z|}{|w|}$ $(w \neq 0)$
  \end{enumerate}
\end{satz}

\begin{bemerkung}{Dreiecksungleichung}{S. 12}
  Sei $z, w \in \C$.
  Dann gilt (wie im Reellen) für den Betrag im Komplexen die \begriff{Dreiecksungleichung}:
  \begin{align}
    | z \pm w | \leq |z| + |w|
  \end{align}
  Allgemein gilt:
  \begin{align}
    \left| \sum_{k=1}^n z_k \right| \leq \sum_{k=1}^n z_k \left| z_k \right|
  \end{align}
\end{bemerkung}

\begin{definition}[Kreisscheibe, $r$-Umgebung einer komplexen Zahl]{S. 12}
  \label{def:1_2}
  Es sei $a \in \C$ und $r \in \R$, $r > 0$.
  Dann ist die Menge
  \begin{align}
    K_r (a) \coloneqq \{ z \in \C : \ |z-a| < r \}
  \end{align}
  die \begriff{(offene) Kreisscheibe} um $a$ mit Radius $r>0$ (oder \begriff{$r$-Umgebung} von $a$)
\end{definition}

\begin{bemerkung}{Polarkoordinatendarstellung komplexer Zahlen}{S. 14}
  Eine komplexe Zahl $z = x + i \, y, z = 0$, ist eindeutig bestimmt durch den \begriff{Betrag}
  \begin{align}
    r \coloneqq |z| = \sqrt{x^2 + y^2}
  \end{align}
  und durch das \begriff{Argument}
  \begin{align}
    \operatorname{arg}(z) \coloneqq \varphi, \ (0 \leq \varphi < 2\pi).
  \end{align}
  Es gilt:
  \begin{align}
    x = |z| \cdot \cos\varphi \text{ und } y = |z| \cdot \sin\varphi .
  \end{align}
  Man nennt
  \begin{align}
    z = r \cdot \left( \cos\varphi + i \cdot \sin\varphi \right)
  \end{align}
  die \begriff{Polarkoordinatendarstellung} von $z \neq 0$.
  Es gilt auch:
  \begin{align}
    z = r \cdot \left[ \cos (\varphi + 2 k \pi) + i \cdot \sin (\varphi + 2 k \pi) \right],\ k \in \Z .
  \end{align}
\end{bemerkung}

\begin{bemerkung}{Umrechnung zwischen Darstellungen}{S. 15}
  Für die Umrechnung zwischen den Darstellungen $z = x + i \, y$ und $z = r \cdot \left( \cos\varphi + i \cdot \sin\varphi \right)$ gelten folgende Regeln:
  \begin{enumerate}
    \item Gegeben sei $z = x + i \, y \neq 0$.
      Mit
      \begin{align}
        r &\coloneqq |z| = \sqrt{x^2 + y^2} \text{ und}\\
        \operatorname{arg}(z) &= \varphi \coloneqq
          \begin{cases}
            \arccos \frac{x}{r} & \text{für } y \geq 0\\
            2 \pi - \arccos \frac{x}{r} & \text{für } y < 0
          \end{cases}
      \end{align}
      gilt
      \begin{align}
        z = r \cdot \left( \cos\varphi + i \cdot \sin\varphi \right) .
      \end{align}
    \item Gegeben sei $z = r \cdot \left( \cos\varphi + i \cdot \sin\varphi \right)$ mit $r > 0$, $\varphi \in [o,2\pi)$.
      Mit
      \begin{align}
        x &\coloneqq r \cdot \cos \varphi \text{ und}\\
        y &\coloneqq r \cdot \sin \varphi
      \end{align}
      gilt
      \begin{align}
        z = x + i \, y .
      \end{align}
  \end{enumerate}
\end{bemerkung}

\begin{satz}{S. 16}
\label{satz:1_3}
  Sind $z, w \in \C\setminus\{0\}$ mit
  \begin{align}
    z = |z| \cdot \left( \cos\varphi + i \cdot \sin\varphi \right), \quad w = |w| \cdot \left( \cos\psi + i \cdot \sin\psi \right),
  \end{align}
  so gilt
  \begin{align}
    z \cdot w &= |z| \cdot |w| \cdot \left[ \cos(\varphi + \psi) + i \cdot \sin(\varphi + \psi) \right],\\
    \frac{z}{w} &= \frac{|z|}{|w|} \cdot \left[ \cos(\varphi - \psi) + i \cdot \sin(\varphi - \psi) \right] .
  \end{align}
\end{satz}

\begin{bemerkung}{Folgerung}{S. 16}
  Aus Satz \ref{satz:1_3} folgt:
  \begin{align}
    \operatorname{arg}(z \cdot w) = \operatorname{arg}(z) + \operatorname{arg}(w) \quad (\mod 2 \pi)\\
    \operatorname{arg}\left(\frac{z}{w}\right) = \operatorname{arg}(z) - \operatorname{arg}(w) \quad (\mod 2 \pi)
  \end{align}
\end{bemerkung}



\subsection{Die komplexe Exponentialfunktion (Teil 1)}

\begin{bemerkung}{Reelle Exponentialfunktion}{S. 17}
  Bekannte reelle Exponentialfunktion ($x, y \in \R$):
  \begin{align}
    e^x &= \sum_{n=0}^\infty \frac{x^n}{n!},\\
    e^{x+y} &= e^x \cdot e^y
  \end{align}
\end{bemerkung}

\begin{bemerkung}{Komplexe Exponentialfunktion}{S. 17}
  Bekannte Komplexe Exponentialfunktion ($z, w \in \C$):
  \begin{align}
    e^z &\coloneqq \sum_{n=0}^\infty \frac{z^n}{n!},\\
    e^{z+w} &= e^z \cdot e^w \label{eq:potenz_komplex}
  \end{align}
\end{bemerkung}

\begin{bemerkung}{Eulersche Gleichung}{S. 18}
  Für $x, y \in \R$ und $z \coloneqq x + i\,y$ gilt:
  \begin{align}
    e^{iy} &= \cos(y) + i \, \sin(y) \qquad \text{Eulersche Gleichung}\\
    e^z &= e^x \cdot e^{i \, y} = e^x \cdot e^{i \, y} = e^x \cdot \left[ \cos(y) + i \, \sin(y) \right]
  \end{align}
\end{bemerkung}

\begin{bemerkung}{Eulersche Identität}{S. 18}
  Aus der Eulerschen Gleichung folgt sofort die Eulersche Identität:
  {
    \Huge
    \begin{align}
      e^{i\pi} + 1 = 0 \notag
    \end{align}
  }
\end{bemerkung}

\begin{satz}{S. 19}
  \label{satz:1_4}
  \begin{enumerate}
    \item Es gilt für $y \in \R$:
      \begin{align}
        \cos y &= \operatorname{Re}\left( e^{iy} \right) = \frac{e^{iy} + e^{-iy}}{2},\\
        \sin y &= \operatorname{Im}\left( e^{iy} \right) = \frac{e^{iy} - e^{-iy}}{2i}
      \end{align}
    \item Jede komplexe Zahl $c \in \C\setminus\{0\}$ lässt sich in der Form
      \begin{align}
        z = r \cdot e^{i\varphi}
      \end{align}
      mit $r = |z|$ und $\varphi = \operatorname{arg}(z)$ schreiben.
    \item Für $z \cdot r \cdot e^{i\varphi}$, $w = s \cdot e^{i\psi}$ gilt:
      \begin{align}
        z \cdot w &= r \cdot s \cdot e^{\varphi + \psi}\\
        \frac{z}{w} &= \frac{r}{s} \cdot e^{\varphi - \psi} \ (w \neq 0).
      \end{align}
  \end{enumerate}
\end{satz}

\begin{satz}[Formel von Moivre]{S. 19}
  \label{satz:1_5}
  Für $n \in \N_0$ gilt:
  \begin{align}
    \left( \cos \varphi + i \, \sin \varphi \right)^n = \cos \left( n \, \varphi \right) + i \, \sin \left( n \, \varphi \right)
  \end{align}
\end{satz}



\subsection{Punktmengen in der komplexen Ebene}

\begin{definition}[Gebiet]{S.20}
  Eine Teilmenge $G \subseteq \C$ heißt \begriff{Gebiet}, wenn $G$ offen und zusammenhängen ist.
\end{definition}

\begin{definition}[einfach zusammenhängend]{S.21}
  Ein Gebiet $G \subseteq \C$ heißt \begriff{einfach zusammenhängend}, wenn das Innere jedes in $G$ verlaufenden geschlossenen Streckenzuges ganz zu $G$ gehört, d.h. wenn $G$ keine Löcher hat.
\end{definition}

\begin{bemerkung}{Randpunkt, Rand von $D$}{S. 21}
  Ist $D \subseteq \C$ eine beliebige Menge, so heißt ein Punkt $z \in \C$ ein \begriff{Randpunkt} von $D$, wenn in jeder $r$-Umgebung (Def. \ref{def:1_2} S. \pageref{def:1_2}) von $z$ sowohl Punkte aus $D$ liegen, als auch Punkte, die nicht zu $D$ gehören.
  Der \begriff{Rand von $D$} ist die Menge aller Randpunkte; er wird mit
  \begin{align}
    \partial D
  \end{align}
  bezeichnet.
\end{bemerkung}

\begin{bemerkung}{abgeschlossen}{S. 22}
  Eine Menge $D \subseteq \C$ heißt \begriff{abgeschlossen}, wenn der Rand von $D$ zu $D$ gehört: $\partial D \subseteq D$.
  Man nennt $\overline{D} \coloneqq D \cup \partial D$ den \begriff{Abschluss} von $D$.
\end{bemerkung}

\begin{bemerkung}{beschränkt, unbeschränkt}{S. 22}
  Eine Menge $D \subseteq \C$ kann beschränkt oder unbeschränkt sein.
  Wir nennen $D$ \begriff{beschränkt}, wenn es eine (hinreichend große) Kreisscheibe $K_r(0)$ gibt, die $D$ umfasst.
  Andernfalls heißt $D$ \begriff{unbeschränkt}.
\end{bemerkung}



\subsection{Zahlenfolgen in der komplexen Ebene}

\begin{definition}[konvergente Folge]{S. 22}
  Man sagt, eine komplexe Zahlenfolge $\left( z_n \right)_{n \geq 0}$ \begriff{konvergiert} gegen den Grenzwert $z \in \C$, und man schreibt
  \begin{align}
    \lim_{n \to \infty} z_n = z \quad \text{ oder } \quad z_n \to z ,
  \end{align}
  wenn es zu jedem $\varepsilon > 0$ einen Index $N(\varepsilon) \in \N_0$ gibt, so dass gilt:
  \begin{align}
    | z_n -z | < \varepsilon \quad \text{ für alle } n \in N(\varepsilon) .
  \end{align}
  Konvergiert die Folge nicht, so nennt man sie \begriff{divergent}.
\end{definition}

\begin{bemerkung}{$z_n\to\infty$}{S. 23}
  Wir schreiben $\displaystyle\lim_{n \to \infty} z_n = \infty$, falls für die \textbf{reelle} Folge $\left( |z_n| \right)$ gilt:\\
  \begin{align}
    \displaystyle\lim_{n \to \infty} |z_n| = +\infty . \notag
  \end{align}
\end{bemerkung}

\begin{bemerkung}{Rechenregeln konvergenter komplexer Folgen}{S. 23}
  Wie bei reellen Folgen ist der Grenzwert einer konvergenten komplexen Folge eindeutig bestimmt und es gelten die bekannten Rechenregeln für $z, w \in \C$:
  \begin{align}
    \begin{rcases}
      \displaystyle \lim_{n \to \infty} z_n = z\\
      \displaystyle \lim_{n \to \infty} w_n = w
    \end{rcases}
    \Rightarrow
    \begin{cases}
      \displaystyle \lim_{n \to \infty} \left( z_n \pm w_n \right) &= z \pm w \\
      \displaystyle \lim_{n \to \infty} \left( z_n \cdot w_n \right) &= z \cdot w \\
      \displaystyle \lim_{n \to \infty} \frac{z_n}{w_n} &= \displaystyle \frac{z}{w} \ (w_n, w \neq 0) \\
      \displaystyle \lim_{n \to \infty} |z_n| &= |z|
    \end{cases}
  \end{align}
\end{bemerkung}
